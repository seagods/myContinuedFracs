\documentclass[12pt]{article}




\setlength{\textheight}{220.0mm}
\setlength{\textwidth}{150.0mm}
\setlength{\oddsidemargin}{0.0mm}
\setlength{\evensidemargin}{20.0mm}
\setlength{\parskip}{5.0mm}
\setlength{\parindent}{5.0mm}
\setlength{\unitlength}{1.0mm}

\newtheorem{defn}{Definition}
\renewcommand{\baselinestretch}{1.2}

\font\bgreek = cmmib10 scaled \magstep1
\def\gammavec{\hbox{\bgreek\char'15}}		% Vector gamma.
\def\bargamma{\skew2\tilde{\gammavec}}		% Vector gamma, bar

\def\Xint#1{\mathchoice
{\XXint\displaystyle\textstyle{#1}}%
{\XXint\textstyle\scriptstyle{#1}}%
{\XXint\scriptstyle\scriptscriptstyle{#1}}%
{\XXint\scriptscriptstyle\scriptscriptstyle{#1}}%
\!\int}
\def\XXint#1#2#3{{\setbox0=\hbox{$#1{#2#3}{\int}$ }
\vcenter{\hbox{$#2#3$ }}\kern-.55\wd0}}
\def\ddashint{\Xint=}
\def\dashint{\Xint-}

\usepackage{amsmath}  %% need if only for numberwithin!
\numberwithin{equation}{section}
\numberwithin{figure}{section}

\usepackage{esint}  %% more nice integral signs
%%  Note texlive-esint must be installed for this to work!
\usepackage{amssymb}  %% set of reals etc.
\usepackage{amsfonts} %% need for the set \mathcal{R} etc
\usepackage{braket} %% Dirac bra and ket vectors

%% Here we are using biblatex rather than bibtex
%%\usepackage[backend=bibtex, sorting=none]{biblatex} %Without sorting=none we get alphaphetical

%%\bibliographystyle{prsty}
%%\bibliography{./Vol3.bib}

\usepackage{fancyhdr}  %% copied from Latex Hints and Tips
\pagestyle{fancy}
%% L/C/R denote left/center/right header (or footer) elements
%% E/O denote even/odd pages

%% \leftmark, \rightmark are chapter/section headings generated by the 
\fancyhead[LE,RO]{\slshape\thepage}
\fancyhead[RE]{\slshape \leftmark}
\fancyhead[LO]{\slshape \rightmark}
%%\fancyfoot[LO,LE]{\slshape Short Course on Asymptotics} %% Don't want a footnote
\fancyfoot[C]{}
%%\fancyfoot[RO,RE]{\slshape 7/15/2002}  %% Leave it out!

%% book document class

\usepackage{mathrsfs}


\numberwithin{figure}{section}
%\numberwithin{figure}{subsection}
\numberwithin{equation}{section}
%\numberwithin{equation}{subsection

\newcommand*{\cofrac}[2]{%
  {%
    \rlap{$\dfrac{1}{\phantom{#1}}$}%
    \genfrac{}{}{0pt}{0}{}{#1+#2}%
  }%
}


\newcommand{\K}{\operatornamewithlimits{K}}

\newtheorem{theorem}{Theorem}
\newtheorem{axiom}{Axiom}


\begin{document}

\thispagestyle{empty}

\title{My Continued Fractions at GitHub}

\maketitle

\tableofcontents

\section{Introduction}

Hwere, we shall simply state what continued fractions are and list whatever theorems and results
the code here uses. (I shall put in some decent references later.)
I started off by writing a class for simple fractions, that is fractions
of the form 
\begin{equation}
F=a+\frac{b}{c}=\frac{B}{c}.
\end{equation}
OK, we have overloaded *,/,+, and - operators and so on. The integers $a$, $b$, $c$, and $B$
 are going to be positive in every case. The class has a bool member to give $F$ a sign.
The thing to note here is the integers are Gnu Multiple Precision numbers. That is to say, these numbers
can be arbitrarily large: the are not constrained by 32 or 64 bits to represent them.
We have the option to GetFloat --- get the floating point value. we can set a limit to the number
 of bytes we can use, and JumpToFloat if the integers go over that limit. Once 
that is done, the fraction only has a floating point value. The GMPfrac class
has a bool member called float which is false until JumpToFloat is called. It also
has a member to represent the floating point value.

So, the integers are members of Gnu Multiple Precision's {mpz\_class}. (These are signed integers).
The rationals $\frac{a}{b}$  and $\frac{B}{c}$ are handled using the GMP {mpq\_class}, and the
floating point numbers are handled using the GMP {mpf\_class}. The thing about the GMP rationals, is
 that the results of all arithmetic operations are automatically reduced using successive applications
of  the Euclidean Algorithm (which computes the gratest common factor of two integers).
It might turn out that we don't really need the fractions class that I wrote, but I have included it
in MyLib (https://github.com/seagods/MyLib.git). Anyhow, what do we mean by a continued fraction?

\section{What is a Continued Fraction}

A famous  example of an {\it infinitely} continued fraction is 
\begin{equation}
\sqrt{2}=CF_{\inf}=1+\cfrac{1}{2+\cfrac{1}{2+\cfrac{1}{2+\cfrac{1}{2+\ldots}}}},
\end{equation}
In practice, we can make this a finite fraction by just stopping at some point.
We can define the following as $convergents$ of the continued fraction, namely
the finite continued fractions
\begin{equation}
CF_{0}=1, \> \> CF_{1}=1+\frac{1}{2},
$$   $$
CF_{2}=1+\cfrac{1}{2+\cfrac{1}{2}},
$$   $$
CF_{3}=1+\cfrac{1}{2+\cfrac{1}{2+\cfrac{1}{2}}},
\label{Sea3}
\end{equation}
and so on.  Clearly
\begin{equation}
CF_{3}=1+\cfrac{1}{2+\cfrac{2}{5} }=1+\frac{5}{12}=\frac{17}{12}.
\end{equation}
Any finite continued fraction is a rational number. The next thing to note is that we can represent
the same continued fraction in infinitely many ways! Suppose we have some positive or negative 
intager  $p$. Looking at $CF_3$, we see that
\begin{equation}
CF_{3}=1+\cfrac{(p/p)}{2+\cfrac{1}{2+\cfrac{1}{2}}}=1+\cfrac{p}{2p+\cfrac{p}{2+\cfrac{1}{2}}}.
\end{equation}
Similarly, we can introduce another non zero integer $q$, and
\begin{equation}
CF_{3}=1+\cfrac{p}{2p+\cfrac{p (q/q)}{2+\cfrac{1}{2}}}1+\cfrac{p}{2p+\cfrac{p q}{2q+\cfrac{q}{2}}}.
\end{equation}
So given any rational, we may express it as a finite continued fraction in an infinity of ways.

\subsection{Notation for a General Continued Fraction}

Obviously, this way of writing things down get's out of hand as we get to $CF_{50}$, and a more compact
notation is necessary. We write our general contined fraction using integers $a$ and $b$, and put
\begin{equation}
CF_{\inf}=b_0+\cfrac{a_1}{b_1+\cfrac{a_2}{b_2+\cfrac{a_3}{b_3+\cfrac{b_4}{c_4+\ldots}}}}.
\end{equation}
Two common notations are to write
\begin{equation}
CF_{\infty}=b_0 +\frac{a_1 \vert}{\vert b_1}+\frac{a_2 \vert}{\vert b_2}
+\frac{a_3 \vert}{\vert b_3}+\ldots
=b_0+\K_{i=1}^\infty \frac{a_i}{b_i}.
\end{equation}
the $K$ in the second representation comes from the German "Ketternbruch" which translates as
"chain fraction". We can represent our $CF_3$ for $\sqrt{2}$ as
\begin{equation}
CF_{3}=1 +\frac{1 \vert}{\vert 2}+\frac{1 \vert}{\vert 2}
+\frac{1 \vert}{\vert 2}
=1+\K_{i=1}^3 \frac{1}{2}.
\end{equation}


\section{How do we Compute Continued Fractions}


We start with the finite fraction
\begin{equation}
CF_{3}=b_0 +\cfrac{a_1}{b_1+\cfrac{a_2}{b_2+\cfrac{a_3}{b_3}}}.
\end{equation}
One we we might proceed is to use our GMPfrac class to compute
\begin{equation}
F_1=b_2+\cfrac{a_3}{b_3},
$$ then  $$
F_2 = a_2/F_1,
$$  and $$
F_3=b_1+F_2, \> \> F_4=a_1/F_3, \> \> {\rm and} \> \> CF_3=b_0+F_4.
\end{equation}









\end{document}
